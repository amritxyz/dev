\documentclass[aspectratio=169]{beamer}
\usepackage[utf8]{inputenc}
\usepackage{amsmath}
\usepackage{enumitem}
\usepackage{longtable}

\title{CASE Tools}
\author{Amrit Bhattarai\\
Anupam Baral
}

\begin{document}

\maketitle

\section*{Introduction}
"CASE tools" stands for \textbf{Computer-Aided Software Engineering} tools. These are software applications that provide support for software development and engineering processes. Their main goal is to improve productivity, quality, and consistency in software projects.

\section*{Types of CASE Tools}
CASE tools are often categorized based on the phase of the software development lifecycle they support.

\subsection*{1. Upper CASE Tools}
Used in the \textit{early stages} of development (planning, analysis, design).

\begin{itemize}
    \item Requirement analysis tools
    \item UML modeling tools
    \item ER diagram tools
    \item Project planning tools (like Gantt charts)
\end{itemize}

\subsection*{2. Lower CASE Tools}
Used in the \textit{later stages} (implementation, testing, maintenance).

\begin{itemize}
    \item Code generators
    \item Debuggers
    \item Testing tools
    \item Configuration management systems
\end{itemize}

\subsection*{3. Integrated CASE Tools}
Support the \textit{entire software development lifecycle}.

\begin{itemize}
    \item Rational Rose
    \item Enterprise Architect
    \item Visual Paradigm
    \item IBM Rational Software Architect
\end{itemize}

\section*{Examples of Popular CASE Tools}

\begin{longtable}{|l|l|}
\hline
\textbf{Tool Name} & \textbf{Purpose} \\
\hline
\endfirsthead
\hline
\textbf{Tool Name} & \textbf{Purpose} \\
\hline
\endhead
Lucidchart & Diagramming and modeling (UML, ERD) \\
StarUML & UML modeling and software design \\
Visual Paradigm & Full SDLC support (analysis to deployment) \\
Jira & Project tracking and issue management \\
Git & Version control (lower CASE tool) \\
Selenium & Automated software testing \\
Eclipse IDE & Coding and debugging support \\
\hline
\end{longtable}

\section*{Benefits of CASE Tools}

\begin{itemize}
    \item Improved productivity and efficiency
    \item Better documentation
    \item Enhanced software quality
    \item Support for team collaboration
    \item Reduction in manual errors
\end{itemize}

\section*{Conclusion}
If you're interested in using CASE tools for a specific project or part of the software development cycle (e.g., modeling, testing), let me know — I can suggest tailored tools or help you pick the best one.

\end{document}
